\documentclass[journal]{IEEEtran}%
\usepackage[T1]{fontenc}%
\usepackage[utf8]{inputenc}%
\usepackage{lastpage}%
%
\title{Translate to English using the YNMT natural language translation model}%
\author{***, ***}%
%
\begin{document}%
\normalsize%
\maketitle%
\begin{abstract}
Hyperspectral images with high spatial resolution are of great significance in target classification, change detection and other fields. However, due to the limitations of imaging sensor technology and other factors, it is impossible to directly acquire images with both high spatial resolution and high spectral resolution. Therefore, image fusion becomes a way to acquire high-resolution images. However, hyperspectral image fusion is an ill-posed inverse problem, so a fusion model based on hyperspectral image and multispectral image is proposed by introducing low-rank kernel tensor and smoothing regularization on the basis of tensor decomposition model. Finally, hyperspectral image with high spatial resolution is reconstructed. In order to make full use of the rich spatial information of multi-spectral images, the local spatial information of remote sensing images is captured by constructing spatial maps of multi-spectral images, and the smoothness between adjacent spectra is promoted by using the spectral regularization based on proximity. Finally, an efficient solution method is designed by using alternate direction multiplier method (ADMM), which reduces the computational complexity of the algorithm. The experimental results show that the fusion method can reconstruct super resolution hyperspectral image and suppress noise effectively.
\end{abstract}%
\section{Introduction}%
\label{sec:Introduction}%
Remote sensing image is used to collect the electromagnetic wave radiated by earth surface objects from space through sensors, which is widely used in the field of classification and recognition of ground objects. Because different substances have different spectral characteristics, the spectral information of the collected image can be compared with the known endmember characteristics to identify the species of substances in the image. According to the difference of imaging modes, common remote sensing images can be divided into panchromatic images, multispectral images and hyperspectral images. Due to the restrictive relationship between spatial resolution and spectral resolution of remote sensing images, panchromatic images have the highest spatial resolution, but only one spectral band mainly covers the whole visible region. Multispectral images usually contain several to a dozen bands, so their spectral resolution is higher than panchromatic images, but their spatial resolution is slightly lower than panchromatic images. Hyperspectral image has richer spectral information than multispectral image, usually can reach tens to hundreds of bands, but the spatial resolution is the lowest. How to improve the spatial resolution and performance of hyperspectral images is an urgent problem to be solved. Using hyperspectral images and multispectral images to reconstruct high{-}resolution images is an effective way, and it has become a research hotspot in the field of remote sensing.%
\par%
Hyperspectral image super resolution methods include single sensor super resolution, subpixel mapping and multi{-}sensor super resolution. Fusion of hyperspectral and multispectral images is a difficulty in multi{-}sensor super resolution methods. From the perspective of model, this method can be divided into the following three categories: (1) Matrix decomposition based method, which is characterized by transforming the original cube data into a matrix, and then using the corresponding algorithm to extract spectral features and abundance features from the observed image data, and then reconstruct the high{-}resolution image. However, expanding hyperspectral data in three{-}dimensional space to two{-}dimensional space will destroy the structure information of the original data, and then appear spatial distortion and spectral distortion phenomenon. (2) Feature extraction of hyperspectral and multispectral image data is carried out by building a deep neural network model based on the deep learning method. Finally, high{-}resolution images are reconstructed by using the extracted features. For example, Hu et al. introduced the attention module into the hyperspectral fusion problem and established a space{-}spectral attention network to realize the fusion of hyperspectral and multispectral. Theoretically, the quality of reconstructed images can be improved as long as the depth of the network is continuously increased. However, with the increase of the depth of the network, the complexity of the network will continue to rise, and the network will be more difficult to train. Moreover, due to the generalization of deep neural networks, the performance of the network will often fluctuate greatly on different data sets. (3) The method based on tensor decomposition can not only retain the structural information of the original data, but also has strong generalization. The common forms of tensor decomposition are CP decomposition and Tucker decomposition. Li et al. proposed a coupled tensor decomposition model based on sparse kernel tensors. Ma et al. proposed a fusion model based on Tucker decomposition, in which low{-}rank regularization was introduced into the factor matrix to suppress noise interference. Since Tucker decomposition is based on 3D data cube, there is no structural information loss problem, so it has become one of the most concerned tensor decomposition models.%
\par%
In this paper, Tucker decomposition is carried out on hyperspectral and multispectral data by comprehensive use of the structural features of the tensor, and logarithmic low{-}rank sum regularization is carried out on the decomposed nuclear tensor and factor matrix respectively. The spatial resolution of hyperspectral data is improved by the local spatial similarity of multispectral images and the correlation of adjacent spectral features, so as to enhance the quality of hyperspectral super{-}resolution images. Finally, using alternating optimization method (AO) and alternating direction multiplier method (ADMM), a set of efficient solutions are designed for the proposed hyperspectral image super resolution model (MLGR) based on multi{-}mode low{-}rank and graph regularity. Experimental results on several widely used open source data sets show that MLGR algorithm can effectively construct super resolution images.%
\par

%
\end{document}