\documentclass[journal]{IEEEtran}%
\usepackage[T1]{fontenc}%
\usepackage[utf8]{inputenc}%
\usepackage{lastpage}%
%
\title{Translate to English using GPT3.5}%
\author{****, ****}%
%
\begin{document}%
\normalsize%
\maketitle%
\begin{abstract}


High-spatial-resolution hyperspectral images have significant importance in target classification, change detection, and other fields. However, due to limitations in imaging sensor technology and other factors, it is not possible to directly obtain images with both high spatial and spectral resolution. Therefore, image fusion has become a way to obtain high-resolution images. However, hyperspectral image fusion is an ill-posed inverse problem. To address this, a fusion model based on hyperspectral and multispectral images is proposed, using tensor decomposition with kernel tensor low-rank and graph smoothness regularization. This model reconstructs high-spatial-resolution hyperspectral images. To fully utilize the spatial information in multispectral images, a spatial map is constructed to capture local spatial information in remote sensing images. Additionally, a spectral regularization based on neighboring spectra is used to promote smoothness between adjacent spectra. Finally, an efficient solution method is designed using the alternating direction method of multipliers (ADMM) to reduce the computational complexity of the algorithm. Experimental results with various remote sensing data sets show that this fusion method can reconstruct hyperspectral images with super-resolution and effectively suppress noise.
\end{abstract}%
\section{Introduction}%
\label{sec:Introduction}%
Remote sensing images are collected by sensors that capture electromagnetic radiation from objects on the Earth's surface in space, and are widely used in fields such as land classification and identification. Because different materials have different spectral characteristics, the type of material in the image can be identified by comparing the spectral information of the collected image with known endmember features. Depending on the imaging mode, common remote sensing images can be divided into panchromatic images, multispectral images, and hyperspectral images. Due to the constraint relationship between spatial resolution and spectral resolution of remote sensing images, panchromatic images have the highest spatial resolution, but only one band mainly covers the entire visible light region; multispectral images usually contain several to tens of bands, so their spectral resolution is higher than that of panchromatic images, but their spatial resolution is slightly lower than that of panchromatic images; hyperspectral images have richer spectral information than multispectral images, usually ranging from tens to hundreds of bands, but their spatial resolution is the lowest. How to improve the spatial resolution and performance of hyperspectral images is an urgent problem, and using hyperspectral and multispectral images to reconstruct high{-}resolution images is an effective way, and has become a research hotspot in the field of remote sensing.%
\par%
The hyperspectral image super{-}resolution method includes single{-}sensor super{-}resolution, sub{-}pixel mapping, and multi{-}sensor super{-}resolution. Based on the fusion of hyperspectral and multispectral images, multi{-}sensor super{-}resolution methods can be divided into three categories from a model perspective: (1) matrix decomposition{-}based methods, which convert the original cube data into a matrix and extract spectral and abundance features from the observed image data using corresponding algorithms to reconstruct high{-}resolution images. However, unfolding the hyperspectral data in three{-}dimensional space into two{-}dimensional space can destroy the original data's structural information, resulting in spatial distortion and spectral distortion. (2) Deep learning{-}based methods extract features from hyperspectral and multispectral image data by building deep neural network models and ultimately reconstruct high{-}resolution images using the extracted features. For example, Hu et al. introduced an attention module into the hyperspectral fusion problem and established a spatial{-}spectral attention network to achieve hyperspectral and multispectral fusion. Theoretically, as long as the network's depth continues to increase, the quality of the reconstructed image can be improved. However, as the network's depth increases, the network's complexity will continue to rise, making it more challenging to train the network. Moreover, due to the generalization problem of deep neural networks, the network's performance often fluctuates significantly on different datasets. (3) Tensor decomposition{-}based methods can preserve the original data's structural information and have strong generalization ability. Common forms of tensor decomposition include CP decomposition and Tucker decomposition. Li et al. proposed a coupled tensor decomposition model based on sparse kernel tensors. Ma et al. proposed a fusion model based on Tucker decomposition and introduced low{-}rank regularization to suppress noise interference in the factor matrix. Since Tucker decomposition is based on three{-}dimensional data cubes and there is no structural information loss problem, it has become one of the most popular tensor decomposition models.%
\par%
This article utilizes the structural characteristics of tensors to perform Tucker decomposition on hyperspectral and multispectral data. The logarithmic low{-}rank and graph regularization are applied to the decomposed core tensor and factor matrices respectively. The local spatial similarity of multispectral images and the correlation of adjacent spectral features are used to improve the spatial resolution of hyperspectral data, thereby enhancing the quality of hyperspectral super{-}resolution images. Finally, an efficient set of solution methods based on alternating optimization (AO) and alternating direction method of multipliers (ADMM) are designed for the proposed hyperspectral image super{-}resolution model (MLGR) based on multi{-}mode low{-}rank and graph regularization. Experimental results on multiple widely used open{-}source datasets demonstrate that the MLGR algorithm can effectively construct super{-}resolution images.%
\par

%
\end{document}